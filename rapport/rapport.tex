\documentclass[11pt]{article}
\usepackage[T1]{fontenc}
\usepackage[utf8]{inputenc}
\usepackage[french]{babel}
\usepackage[margin=2.5cm]{geometry}
\usepackage{graphicx}
\usepackage{mathptmx}
\usepackage{setspace}
\usepackage{titlesec} 
\usepackage{imakeidx} 
\usepackage{tcolorbox}


\setlength{\parindent}{1cm}
\makeindex
\renewcommand{\rmdefault}{ptm}
\onehalfspacing % Interligne de 1.5
\setcounter{page}{1}
\titleformat{\section}
{\fontsize{16}{19}\selectfont\bfseries} 
{\thesection}
{20pt}
{}

\titleformat{\subsection}
{\fontsize{15}{19}\selectfont\bfseries} 
{\thesubsection}
{20pt}
{}


\begin{document}
\pagenumbering{gobble} % Supprime la numérotation des pages
\begin{titlepage}
    \begin{center}

        \begin{minipage}[b]{0.3\textwidth}
        \includegraphics[width=\textwidth]{images/Logo_Universite_de_Lorraine.png}
        \end{minipage}
        \begin{minipage}[b]{0.2\textwidth}
            \centering
            \includegraphics[width=\textwidth]{images/logo-fst-format-jpg-couleur.jpg}
        \end{minipage}
        \smallbreak
        \vspace{0.5cm}
        \textbf{\large Master Informatique}

        %% Milieu de la page
        \vfill
        {\Large Reconnaissance des mouvements de la main} \smallbreak
        Rapport \smallbreak en vue de la validation de l'UE Initiation à la recherche \smallbreak
        \vfill
        \begin{tabular}{ccc|ccc}
            \'Etudiants : & Victor Dallé & & & Encadrante : & Mme. Boltcheva \\
            & Claire Kurth & & & & 
            
        \end{tabular}
    \end{center}

\end{titlepage}

\newpage \newpage
\section*{Décharge de responsabilité }\bigbreak
L'Université de Lorraine n'entend donner ni approbation  ni improbabtion aux opinions émises dans ce rapport,
ces opinions devant être considérées comme propres à leurs auteurs. \bigbreak

\newpage
\section*{Remerciements}

\newpage
\tableofcontents
\newpage

\pagenumbering{arabic}
\setcounter{page}{1}
\section*{Introduction}
\addcontentsline{toc}{section}{Introduction}
La reconnaissance des gestes représente un domaine en plein essor de la vision par ordinateur, incarnant une révolution dans la manière dont les utilisateurs interagissent avec les systèmes informatiques. Cette technologie est en effet en train de transformer la façon dont nous utilisons les machines,
offrant des opportunités novatrices dans des domaines tels que l'interaction entre l'homme et la machine, 
la réalité augmentée ou encore l'accessibilité numérique. \bigbreak

Cette nouvelle manière d'interagir avec un système numérique est déjà utilisée dans plusieurs domaines notamment le sport avec des applications de coaching personnel qui permettent de suivre les mouvements de l'utilisateur et ainsi lui donner des conseils pour améliorer sa technique, ou encore sa posture. Ce nouveau concept d'interraction permet également de pouvoir contrôler des appareils tels que des téléviseurs où par un simple geste, nous pouvons par exemple gérer le son ou changer de chaîne. \bigbreak

Contrairement aux interfaces traditionnelles basées sur le clavier et la souris, la reconnaissance des gestes permet aux utilisateurs de communiquer plus simplement avec les ordinateurs, en utilisant leurs mains ou leur corps pour contrôler les applications, ou encore naviguer dans des environnements virtuels. Cette approche favorise une expérience utilisateur plus immersive et ergonomique, ouvrant ainsi de nouvelles perspectives dans des domaines variés tels que le divertissement interactif, l'éducation, ou encore la médecine. \bigbreak

Comme dit précédemment, la reconnaissance des mouvements joue un rôle crucial dans l’accessibilité numérique en permettant à des personnes ayant un handicap physique  ou moteur de pouvoir communiquer et d'interagir avec des outils numériques plus facilement. En effet, celà permet de passer outre les obstacles liés à l’utilisation des outils traditionnels (tels que le clavier, la souris, la télécommande …) grâce à la simple utilisation de mouvements du corps. \bigbreak

Dans ce contexte, ce projet vise à implémenter un système de reconnaissance des mouvements de la main à l’aide d’un classificateur classique Haar-cascade. 
L'objectif principal est de concevoir un système capable de détecter et de classifier différents gestes de la main effectués par l'utilisateur, 
tels que le poing fermé, un seul doigt levé, deux doigts, trois doigts, quatre doigts, et ainsi de suite. Ces gestes seront ensuite associés à différentes 
actions telles que le lancement d'applications ou encore l'ouverture de sites web. \bigbreak

Pour réaliser ce projet nous utiliserons principalement la bibliothèque OpenCV. Nous verrons ce qui est actuellement fait en termes de reconnaissance des 
gestes, puis nous expliquerons quelques notions et nous verrons ensuite comment nous avons implémenté, dans un premier temps une reconnaissance de la main 
sans un classificateur classique Haar-cascade et dans un second temps avec.

\newpage

\section{Rappel du sujet et encadrement}
\subsection{Rappel du sujet}
\subsection{Encadrement}
\newpage

\section{\'Etat de l'art}
\subsection{Article de départ}
\subsection{Médiapipe}
\subsection{Classificateur classique Haar-cascade}
\newpage

\section{Quelques explications}
\subsection{Thresholding et Convex Hull}
\subsection{Classificateur classique Haar-cascade}
\newpage

\section{Reconnaissance de la main sans classificateur}
\subsection{Méthodologie}
\subsection{Expériences et résultats}
\subsection{Conclusion}
\newpage

\section{Reconnaissance de la main avec classificateur classique Haar-cascade}
\subsection{...}

\section*{Conclusion}
\addcontentsline{toc}{section}{Conclusion}
\newpage

\section*{Annexes}
\addcontentsline{toc}{section}{Annexes}

\section*{Bibliographie}
\addcontentsline{toc}{section}{Bibliographie}
\newpage

\section*{Glossaire}

\section*{Déclaration sur l'honneur contre le plagiat}

\newpage

\newpage
\section*{Résumé}

\end{document}